\documentclass{article}

\usepackage{placeins}

\begin{document}

\begin{titlepage}
		\noindent{\huge Minimum Fill-In} \\ \\
		Mikkel Gaub, \\ Malthe Ettrup Kirkbro, \\ \& Mads Frederik Madsen	\\ \\
		\hspace{-18pt}
		\textit{June 2, 2017}
		\thispagestyle{empty}
	\end{titlepage}
	\clearpage

	\section{Algorithm}
	The algorithm REFREF is concerned with finding four-cycles and certain substructures in the graph, called moplexes, which are defined as adhering to the following rules: \\

	\begin{description}
		\item[Clique] Every vertex in the moplex must be connected to all other vertices.
		\item[Shared neighbourhood] Every vertex in the moplex must have the same neighbourhood.
		\item[Minimal separator neighbourhood] The neighbourhood of the moplex is a minimal separator, meaning that if the neighbourhood is removed from the graph, the graph is split into multiple parts. Removal of any proper subset of the neighbourhood must not split the graph, hence it is minimal.
		\item[Maximally inclusive] If a vertex can be added to a moplex, without it not being a moplex anymore, that vertex must be included.
	\end{description}

	Furthermore, simplicial moplexes are moplexes where the neighbourhood of the moplex is also a clique.

	As an indicator that an edge will at some point need to be connected to another vertex, markings are used in this algorithm. 
	A maximum of two times k markings are allowed, where k is the number of fill edges in the decision problem.

		\subsection{Kernelization}
		The kernelization algorithm used, REFREF, is split into three phases:

		\begin{description}
			\item[Phase 1]
			\item[Phase 2]
			\item[Phase 3]
		\end{description}

		Since phase 3 is dependent on k, it has to be run for each k-value used by the algorithm...

		\subsection{Cases} 
		Once the graph has been kernelized, the algorithm is called with an initial k-value. The algorithm then tries to find a solution for the graph with the initial k-value.
		If no solution with that k-value can be found, k is incremented and the algorithm is run again.
		The algorithm is recursively called with a the k-value and in each iteration the first of six actions are performed, where the criteria of that case is met.
		These cases are, in order:
		\begin{description}
			\item[Four-cycle] If a four-cycle exists, the program branches on both possible resolution of the four-cycle.
			\item[Moplex with marked and unmarked vertices] If a moplex containing marked and unmarked vertices exists in the graph, all vertices of the moplex are marked.
			\item[Simplicial moplex with only unmarked edges] If \emph{any} simplicial moplexes containing only unmarked edges, exist in the graph, they are removed.
			\item[Unmarked moplex with a neighbourhood missing only one edge] If a moplex containing only unmarked vertices and has a neighbourhood which is only missing one edge in order to be a clique, the missing edge is added.
			\item[Only marked moplexes] If all moplexes in the graph are marked, the algorithm fails.
			\item[Any unmarked moplex] If there are any unmarked moplexes in the graph, the program branches into two. In one of the cases, every vertex in the moplex is marked. In the other case, every edge missing in the neighbourhood of the moplex is added.
		\end{description}

		At the start of each iteration, the graph is checked for chordality and if k has reached zero, the algorithm increments k and starts over.

	\section{Implementation challenges}

		\subsection{Data structure}
		To efficiently store the graph in memory an adjacency list is used.
		In practice this is implemented as a vector of binary search tree sets.
		While asserting if two vertices share an edge is possible in constant time using hash sets, the size of the edge sets are generally not large enough to outweigh the ability to compare sets.

		Other graph representations, like the adjacency matrix, were attempted, but as the algorithm relies heavily on iterating vertex neighborhoods, the less efficient edge list access worsened the overall efficiency of the algorithm severely.

		\subsection{Chordality}

		\subsection{Four-cycles}
		As the algorithm branches on chordal completions of four-cycles, an efficient approach to finding these in the graph is needed.
		It must be possible to either find one four-cycle or report that none exist.
		This is accomplished using a depth-first search to a maximum depth of four, checking for lesser cycles along the way.
		If no lesser cycles are present and the starting vertex is found at a depth of four, a four-cycle is reported.
		If no four-cycles are found starting the search in every vertex of the graph, none are present in the graph.
		The time complexity of this approach is $O(VE^4)$ as it is potentially necessary to check every edge in the graph in all four depths for every vertex.

		Yuster and Zwick\cite{finding-even-cycles} describes an algorithm to detect four-cycles in $O(n^2)$ time, but requiring $O(n)$ extra space.
		The algorithm uses an empty adjacency matrix the size of the graph and turns the neighborhood of each vertex in the original graph into a clique in the matrix.
		If an edge is added to the matrix twice, a four-cycle is present and can be extracted in linear time.
		This algorithm was attempted in the C\# prototype but turned out to be much slower than depth-first search, likely due to the memory overhead.
		With an optimal implementation, perhaps using bit-vectors and the AVX instruction set, this algorithm might be worthwhile, but time constraints prevented such attempts.

		\subsection{Moplexes}

		\subsection{V-star}

	\section{Optimizations}
	The algorithm has been implemented in C++ to allow for greater optimization.

		\subsection{Moplex caching}
		A very large part of the algorithm is in locating every moplex in the graph, since this is used in every case of the algorithm, excluding the case where a four-cycle is found.
		Because the moplexes change fairly little in each iteration of the algorithm, these are cached in order to reduce the amount of time wasted on finding already found moplexes.

		It is accomplished by...

		\subsection{Component splitting}
		Since the graph is possibly divided into smaller kernels by the kernelization algorithm, the problem can potentially reduced to several instances with a lower k each, the performance of the program should greatly increase.

		This is accomplished by running the core of the algorithm on each found component of the graph, with an initial k-value of 0, or the k returned by the kernelization, if only one component exists in the graph.

		\subsection{Subgraphs}
		Many functions used by the algorithm requires subgraphs to be considered, such as the kernelization and the deletion of simplicial moplexes.
		Since the algorithm is depth-first, meaning each branch is explored exhaustively before the next branch is considered, the changes made must be easily reversible.
		Since copying or deleting and recreating the graph is expensive, a cheap way to specificy which parts of the graph are currently "active" would be useful.
		This is a accomplished with a boolean value for each vertex, indicating whether or not it should be regarded.

		\subsection{Maximum cardinality search}
		\label{subsec:maximum-cardinality-search}
		As a part of finding chordless cycles in the graph for phase 1 of the kenerlization algorithm, it is recommended in the paper describing the algorithm\cite{kernel} that the maximum cardinality search (MCS) is used.
		MCS is very closely related to the LexBFS algorithm already used for REFREF, with the main difference being that LexBFS stores the processed neighbours by a list of names, where MCS instead maintains the count of processed neighbours.

		\begin{table}[!ht]
			\begin{tabular}{| l | r | r |}
			\hline
											& MCS 		& LexBFS 	\\ \hline
			2000-fully connected subgraph	& 43 s  	& 39 s 		\\ \hline
			1000-fully connected 			& 23 s 		& 23 s 		\\ \hline
			1000-fully connected subgraph	& 4 s   	& 4 s 		\\ \hline
			10000-cycle 					& 4 s   	& 0.161 s 	\\ \hline 
			500-fully connected 			& 2 s   	& 2 s 		\\ \hline 
			500-fully connected subgraph 	& 0.31 s 	& 0.284 s 	\\ \hline 
			\end{tabular}
			\label{table:maximum-cardinality-search}
		\end{table}
		Intuitively, the MCS algorithm should be atleast as fast as the LexBFS algorithm, however we discovered a considerable speed-up when handling large cycles with LexBFS rather than MCS, without losing speed in other cases, as can be seen in our test results in table \ref{table:maximum-cardinality-search}.
		This is most likely due to the details of each implementation and not something we have investigated further.
		We therefore changed the use of MCS to LexBFS in every case, most notably in the kernelizer.

	\section{Failed optimizations}
	In addition to the actual optimizations listed above, several other things were tried.

		\subsection{Set functions}
		As there are some very specific requirements for set functionality in the algorithm, which the default C++ set library does not support, implementation and optimization of this functionality was attempted.
		Set union for two and three sets is used, along with set intersects for two sets.
		Except for a union of three sets, these methods all exist in the C++ library, and the union of three sets, $s_1$, $s_2$ and $s_3$, can be replaced by two unions: $(s_1 \cup s_2) \cup s_3$.
		This does however have some theoretical overhead, in that the unions would not be able to function in-place and would need an auxillary set, due to the structure of the C++ set library.

		It was also not immediately apparent that the C++ implementation was optimal, since elements in a set are contained in a sorted order and unions and intersects can be done in time complexity of the total number of elements in the given sets.
		The optimized set functionality did not, however, have any effect on the running time of the program, but it was decided that they should be kept due to the brevity of their parameters and ease of use in the case of the union of three sets.
		
		\subsection{Graph squashing}
		As the subgraph implementation relies on flags determining if a vertex is part of the graph, vertices and edges not included in the graph must be checked by all graph operations.
		An attempt to improve on this squashes a subgraph into a minimal representation containing only included vertices and edges.
		Metadata about the original graph, like vertex names, must be preserved in the squashed graph, complicating the operation.
		In practice this optimization attempt is also equivalent to cloning the relevant part of the graph whenever a subgraph is needed. This optimization did not improve the overall running time of the algorithm, likely because the overhead of checking subgraph flags is negligible compared to the computation done on the subgraph data.

	\section{Comparison}

	\section{Conclusion}

	\pagebreak
	\addcontentsline{toc}{section}{References}	
	\bibliographystyle{plain}
	\begin{thebibliography}{99}

		\bibitem{kernel}
		Kaplan, H., Shamir, R. and Tarjan, R. E. 
		\textit{Tractability of Parameterized Completion Problems on Chordal, Strongly Chordal, and Proper Interval Graphs}. 
		SIAM J. OF COMPUTING, Vol. 28, No. 5, pp. 1906--1922

		\bibitem{finding-even-cycles}
		Raphael Yuster and Uri Zwick
		\textit{Finding Even Cycles Even Faster}.
		SIAM J. Discrete Math., 10(2), pp. 209--222.

	\end{thebibliography}

	\clearpage

\end{document}